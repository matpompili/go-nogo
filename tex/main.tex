\documentclass[a4paper]{article}
\usepackage[british]{babel}
\usepackage{classicthesis}
\usepackage{geometry}
\geometry{lmargin=3cm,rmargin=3cm,bmargin=3cm,tmargin=3cm}

\usepackage[autostyle,italian=guillemets]{csquotes}
\usepackage[backend=biber]{biblatex}

\addbibresource{db.bib}

% TODO remove in final version
\usepackage{blindtext}

\usepackage{pgfgantt}


% Title Page
\title{
	\huge{\textsc{A Multi-Node Quantum Network\\with Defects in Diamond}}\\
	\vspace{10pt}\Large{Ph.D. proposal}
}
\author{Matteo Pompili}
\date{November 23, 2018}


\begin{document}
\maketitle

%TODO write intro
\section*{Introduction}
A perfect introduction with a nice citation here \cite{Humphreys2018}.
Also \blindtext


\tableofcontents

\section{Research goals}
The goal of my Ph.D. is:\\
\par
\makebox[\textwidth]{\underline{\textsc{Demonstration of quantum applications on a multi-node network.}}}



%TODO breakdown how to do it

\section{The NV centre as a quantum network node}

Quantum networks are expected to deliver definitive security for communication, blind quantum computation, improved clock synchronization and more exotic applications such as connecting far apart telescopes \cite{Wehner2018}.

\section{Genuine remote multipartite entaglement}

\section{Link layer: a proof of concept}

\section{Entanglement teleportation}

\section{Client-Server secure delegation}

\section{Challenges and risks}

\section{Graduate school progress}
\subsection{Courses}
I attended (or I am currently attendeing) the following courses:
\begin{itemize}
	\item Collaboration across disciplines (? GSC) %TODO ask Sandrine!
	\item PhD Starup (2 GSC)
	\item Conversation skills (2 GSC)
	\item Casimir Course - Programming (5 GSC)
	\item Casimir Course - Electronics for Physicists (5 GSC)
	\item QuTech Academy - Quantum Communication and Cryptography (5 GSC)
\end{itemize}

\subsection{Supervision}
I have been supervising Hans K. C. Beukers, a MSc student, since February 2018. Hans has been working on setup improvements and techniques that, if succesful, will increase the lifetime of our memory qubits.

\subsection{Outreach}
As an Early Stage Researcher in the MSCA ITN Spin-NANO, I have to carry out outreach activities regarding my reaserch field to the wider audience. I have currently carried out two outreach activities:
\begin{itemize}
	\item January 2018, Sheffield, UK. Introdution to quantum- and nano-technologies to local high-school students
	\item September 2018, Brussels, BE. Two days stand about quantum technologies at the Europen Researchers Night, EU Parlamentarium, mainly to children between 5 and 10.
\end{itemize}

\section{Ph.D. timeline}

\begin{ganttchart}[
	hgrid,
	vgrid,
	expand chart=\textwidth
	]{1}{12}
	\gantttitle{Proposed Ph.D. Timeline}{12}\\
	\gantttitle{Year 2}{4} \gantttitle{Year 3}{4} \gantttitle{Year 4}{4}\\
	\ganttbar{3-Node entanglement}{1}{2} \ganttnewline
	\ganttbar{Link layer demonstration}{3}{5} \ganttnewline
	\ganttbar{Entanglement teleportation}{6}{7} \ganttnewline
	\ganttbar{Secure delegation}{8}{9} \ganttnewline
	\ganttbar{Thesis writing}{10}{12}
	
\end{ganttchart}


\section*{Acknowledgements}
%TODO thank everybody
\addcontentsline{toc}{section}{Acknowledgements}

\printbibliography[heading=bibintoc]

\end{document}          
