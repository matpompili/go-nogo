% !TeX spellcheck = en_GB
\documentclass[a4paper]{article}
\usepackage[british]{babel}
\usepackage{classicthesis}
\usepackage{geometry}
\geometry{lmargin=3cm,rmargin=3cm,bmargin=3cm,tmargin=3cm}
% ACRONYMS
\usepackage[printonlyused]{acronym}
\acrodef{NV}{Nitrogen-Vacancy}
\acrodef{ESR}{Early Stage Researcher}
\acrodef{ITN}{Innovative Training Network}
\acrodef{MSCA}{Marie Skłodowska-Curie Actions}
% END ACRONYMS

\usepackage{siunitx}

\usepackage[autostyle,italian=guillemets]{csquotes}
\usepackage[backend=biber]{biblatex}

\addbibresource{db.bib}

% TODO remove in final version
\usepackage{blindtext}
\usepackage{todonotes}

\usepackage{pgfgantt}


% Title Page
\title{
	\huge{\textsc{A Multi-Node Quantum Network\\with Defects in Diamond}}\\
	\vspace{10pt}\Large{Ph.D. proposal}
}
\author{Matteo Pompili}
\date{November 23, 2018}


\begin{document}
\maketitle


\section*{Introduction}
\todo{Write a real introduction}A perfect introduction here.
Also \blindtext


\tableofcontents
\newpage
\section{Research goals}
The goal of my Ph.D. is:\\\\
\makebox[\textwidth]{\underline{\textsc{Demonstration of quantum applications on a multi-node network.}}}
\todo{Breakdown how to do it}

\section{The NV centre as a quantum network node}

Quantum networks are expected to deliver definitive security for communication, blind quantum computation, improved clock synchronization and more exotic applications such as connecting far apart telescopes \cite{Wehner2018}.

A node of such a network needs to: 1) generate entangled states with other nodes, 2) manipulate quantum states and 3) store quantum states. The \ac{NV} centre in diamond is a promising candidate to act as node of such a network, as it fulfils all the mentioned requirements. Figure~\ref{fig:nv_summary} summarises the fundamental properties of the \ac{NV} centre.

\begin{figure}
	\missingfigure[figwidth=\textwidth, figheight=0.5\textwidth]{A nice figure to summarise the NV centre properties}
	\caption{The \ac{NV} centre as a quantum network node. a) The \ac{NV} is an atomic defect in diamond with trapped ion-like properties. b) Spin selective optical transitions allow for high-fidelity initialization and single-shot read-out. c) Neighbouring $C^{13}$ atoms can be used as quantum memories. d) Entanglement can be generated among remote \acp{NV}.}
	\label{fig:nv_summary}
\end{figure}

Recent work from our group demonstrated the on-demand generation of remote entanglement between two \ac{NV} centres with rates up to \SI{39}{\Hz} \cite{Humphreys2018}. Such high rates are a consequence of moving from a two-photon detection protocol, such as the one used in Ref.~\cite{Hensen2015}, to a single-photon protocol.
\section{Genuine remote multipartite entanglement}

\section{Link layer: a proof of concept}

\section{Entanglement teleportation}

\section{Client-Server secure delegation}

\section{Challenges and risks}

\section{Graduate school progress}
\subsection{Courses}
I attended (or I am currently attending) the following courses:
\begin{itemize}
	\item Collaboration across disciplines (? GSC) \todo{Ask Sandrine!}
	\item PhD Start-up (2 GSC)
	\item Conversation skills (2 GSC)
	\item Casimir Course - Programming (5 GSC)
	\item Casimir Course - Electronics for Physicists (5 GSC)
	\item QuTech Academy - Quantum Communication and Cryptography (5 GSC)
\end{itemize}

\subsection{Supervision}
I have been supervising Hans K. C. Beukers, a MSc student, since February 2018. Hans has been working on setup improvements and techniques that, if successful, will increase the lifetime of our memory qubits.

\subsection{Outreach}
As an \ac{ESR} in the \ac{MSCA} \ac{ITN} Spin-NANO, I have to carry out outreach activities regarding my research field to the wider audience. I have currently carried out two outreach activities:
\begin{itemize}
	\item January 2018, Sheffield, UK. Introduction to quantum- and nano-technologies to local high-school students, as part of an \ac{ITN} meeting. \todo{1 GSC?}
	\item September 2018, Brussels, BE. Two days stand about quantum technologies at the European Researchers Night, EU Parlamentarium, mainly to children between 5 and 10. \todo{2 GSC?}
\end{itemize}

\section{Ph.D. time-line}
\begin{figure}[htb!]
	\begin{center}
		\begin{ganttchart}[
			hgrid,
			vgrid,
			expand chart=\textwidth
			]{1}{12}
			\gantttitle{Year 2}{4} \gantttitle{Year 3}{4} \gantttitle{Year 4}{4}\\
			\ganttbar{3-Node entanglement}{1}{2} \ganttnewline
			\ganttbar{Link layer demonstration}{3}{5} \ganttnewline
			\ganttbar[
			bar/.append style={fill=black!10,pattern=north east lines}
			]{Support building $4^{th}$ setup}{3}{5} \ganttnewline
			\ganttbar{Entanglement teleportation}{6}{8} \ganttnewline
			\ganttbar{Secure delegation}{9}{10} \ganttnewline
			\ganttbar{Thesis writing}{11}{12}
		\end{ganttchart}
	\end{center}
	\caption{Proposed Ph.D. time-line. \todo[inline]{Write caption. Explain 4th setup.}}
	\label{fig:phdtimeline}
\end{figure}

\newpage
\section*{Acknowledgements}
\addcontentsline{toc}{section}{Acknowledgements}
\todo{actually thank people}
I would like to thank everybody.

\printbibliography[heading=bibintoc]

\end{document}          
